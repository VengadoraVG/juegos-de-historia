\begin{section}{Alcances}
  Sólamente se desarrollará un nivel del videojuego. Si el tiempo alcanza

  \begin{subsection}{Mundo}
    \begin{itemize}
    \item un tilemap incorporado al juego.
    \item Omar Quispe puede ser controlado con el teclado.
      \begin{itemize}
      \item sprint
      \item caminar
      \item física y colisiones con el piso.
      \end{itemize}
    \end{itemize}
  \end{subsection}

  \begin{subsection}{Medio ambiente}
    \begin{itemize}
    \item cuerpos para lootear
    \item lugares para cubrirse
    \item items para lootear en los cuerpos
    \item los items pueden equiparse
    \item la comida se gasta al usarse
    \item la comida tiene un cooldown y un cast time
    \item el agua se gasta al usarse
    \item el agua tiene un cooldown y un cast time
    \end{itemize}
  \end{subsection}

  \begin{subsection}{Sistema de preguntas y respuestas}
    \begin{itemize}
    \item lugares que dan preguntas
    \item lugares que dan respuestas
    \item seleccionar una pregunta
    \item menú para explorar preguntas y respuestas encontradas
    \item responder una pregunta
    \end{itemize}
  \end{subsection}

  \begin{subsection}{Atributos}
    \begin{itemize}
    \item Omar Quispe tiene vida y energía.
    \item La energía se gasta cuando corre.
    \item La energía se recupera (OT) cuando toma agua.
    \item La vida se recupera (OT) cuando come una ración.
    \end{itemize}
  \end{subsection}

  \begin{subsection}{Armas}
    \begin{itemize}
    \item picota, lanza, escopeta, oz.
    \item tienen un cooldown y un cast time.
    \item la escopeta se puede recargar sin usar inmediatamente.
    \item tienen un área de daño.
    \item tienen puntos de ataque.
    \item tienen una probabilidad de que el ataque sea bloqueado.
    \end{itemize}
  \end{subsection}

  \begin{subsection}{Enemigos}
    \begin{itemize}
    \item artilleros, milicia, soldado, caballería
    \item cada uno con inteligencia artificial primitiva
    \end{itemize}
  \end{subsection}

  \begin{subsection}{Script (historia)}
    \begin{itemize}
    \item El código soporta un script ingame con animaciones en los sprites.
    \item el script inicial de Ladislao Cabrera.
    \item el jugador debe tener la opción de saltarse la escena.
    \item el script final
    \end{itemize}
  \end{subsection}
\end{section}
