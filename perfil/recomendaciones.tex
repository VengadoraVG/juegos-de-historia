Al hacer un videojuego educativo que difunda la historia de una forma entretenida, se pretende que sus mecanismos de juego sean reusables en otros videojuegos que toquen otros aspectos históricos. 

Al hacer un videojuego sobre la guerra del pacífico, y presentarlo en los canales adecuados, se hará difusión sobre la demanda marítima boliviana, tanto a nivel nacional, como a nivel internacional. 

Con el conocimiento y herramientas adquiridas como resultado de la culminación de este proyecto, se pretende sentar un precedente para el estudio formal de la creación de videojuegos en la carrera de informática, de la Universidad Mayor de San Andrés. 

Luego de culminar el proyecto, con los datos recopilados de los jugadores, se pretende sentar un precedente para una investigación en el área de psicología, en el campo del aprendizaje, para determinar si el juego enseña historia de forma efectiva, y los factores que causan esto.
